\subsection{Hierarchical Abstract Machines}

Hierarchical Abstract Machines (HAM) \cite{Parr} use a hierarchy of non-deterministic finite state machines, each defined by a partial policy.
HAMs provide a flexible way of limiting the number of actions an agent can take at each state.
A machine is defined by an initial state determined by a start function, a set of states $<Action,\ Call,\ Choice,\ Stop>$, and a transition function.
%The states are: Action, Call, Choice and Stop.
An action state generates
a primitive MDP action. The call state halts the
execution of the current machine and calls another
machine. The choice state non-deterministically selects the next state, governed by the transition function. The
stop state terminates the execution of the current machine and returns it to the previous call state. The HAM setup includes one start machine and all machines reachable from the initial state of this machine. This hierarchical setup allows for SMDP Q-learning. 

Theoretically, it is the most flexible framework. Options can be represented
by HAMs by omitting the call state, but all HAMs cannot be represented by
options unless one option's policy allows for other options to be called. We
have not encountered such a case in our study. They can be extended to Partially Observable MDPs (POMDP)
by only using the choice state of the machine when the agent is completely
certain of the current state in the POMDP. It is unclear whether they can be
applied to POMDPs when such states do not exist. HAMs work because they
constrain the policy at each state, effectively reducing the state space.
Poorly defined machines will reduce the state space to a point where the
original problem is no longer meaningful.
