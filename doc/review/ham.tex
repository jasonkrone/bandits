\section{Hierarchical Abstract Machines}

HAM brings about the concept of machines. A machine is a partial policy
represented by a Finite State automaton. Such a machine has multiple states
and actions, providing both state and temporal abstraction with learning occuring
within machines, as they are only partially defined. Rather than expanding
action choices like in options, HAM restricts the size of
the class of policies across multiple layers, allowing HAM policies
to be specified as  hierarchies of stochastic finite-state machines.

In a HAM, the entire system is considered as a collection of stateful machines.
Possible states are: Action, Call, Choice and Stop. An action state generates
an primitive MDP acation. The call state halts the
execution of the current machine and calls another
machine. The choice state non-deterministically selects the next state. The
stop state terminates the execution of the current machine \cite{Parr}. When
the initial state has actions with no
call states, then this acts as a solution to basic flat RL. Being
hierarchical, SMDP Q-learning can be applied to this set-up as well.

\textbf{Strengths}
\begin{itemize}
    \item Theoretically, it is the most flexible framework allowing multiple inputs like policy, task hierarchy and effort required. Note that except for the policy, other inputs are optional
    \item Can be extended to partial programs
    \item Has lesser stringent requirements than options
    \item Allows concurrency by letting simultaneous functioning of multiple machines
    \item Abstraction can be embedded into the system using three value decomposition
    \item Provides a richer representation and concurrency
\end{itemize}

\textbf{Weaknesses}
\begin{itemize}
    \item Needs a proper mechanism for designing the machines
    \item Requires a complete policy
    \item No abstraction is provided in the original specification
\end{itemize}

\textbf{Flexibility}\\
HAMs can easily be reduced to many other hierarchical reinforcement learning
frameworks, making them one of the more flexible options available.
