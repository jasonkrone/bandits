\subsection{Feudal Reinforcement Learning}
The feudal RL \cite{Dayan} hierarchy consists of managers and sub-managers.
At each level of the hierarchy the manager in control can either perform
a primitive action or pass control to a sub-manager.
When control is passed to a sub-manager, the manager must wait until
the state changes at the managerial level before they can act again.

This approach relies on two forms of abstraction: 1. Reward Hiding 2. Information Hiding.
Reward Hiding requires that managers reward sub-managers for achieving the goals they set.
This is the case even if the actions taken by the sub-managers are of detriment to the goals of the manager.
Reward Hiding enables low level sub-managers to learn from accomplishing sub-tasks early in training
because the success of these sub-managers is not dependent on accomplishing the highest level goal.
Information Hiding requires that sub-managers only know information about the state of the system
that is required to accomplish the task set by their manager. Therefore, the task of each agent (manager or sub-manager)
is only known to that agent. Additionally, managers must know the satisfaction conditions for tasks they
set for sub-managers so that they can reward them appropriately.

Like other hierarchical methods, the feudal structure decomposes state-space such that each agent is
only responsible for operating over a subset of all states. This makes it easier for each agent to
learn a value function over their states and actions, which in turn makes it easier for agents to learn
an optimal policy. Additionally, the hierarchy allows for greater control over exploration. Specifically,
broad exploration can be executed at high level in the hierarchy and more granular exploration from low levels in the
hierarchy.

One drawback of feudal RL is that it assumes there is a natural hierarchical division of the environment and, ideally,
the state space. In addition, Information Hiding can create inefficiency by teaching sub-managers  to satisfy tasks
that are orthogonal to accomplishing the highest level goal.


